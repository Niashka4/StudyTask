Человек с тонкой шеей забрался в сундук, закрыл за собой крышку и начал задыхаться.

"--* Вот,"--- говорил, задыхаясь человек с тонкой шеей,"--- я задыхаюсь в сундуке, потому что у меня тонкая шея. Крышка сундука закрыта и не пускает ко мне воздуха. Я буду задыхаться, но крышку сундука все равно не открою. Постепенно я буду умирать. Я увижу борьбу жизни и смерти. Бой произойдет неестественный,  при равных шансах, потому что естественно  побеждает смерть, а жизнь, обреченная на смерть, только тщетно борется с врагом, до последней минуты не теряя напрасной надежды. В этой же  борьбе, которая произойдет сейчас, жизнь  будет знать способ своей победы: для этого жизни надо заставить мои руки открыть крышку сундука. Посмотрим: кто кого? Только вот ужасно пахнет нафталином. Если победит жизнь, я буду вещи в  сундуке пересыпать  махоркой\dots Вот началось: я больше не могу дышать. Я погиб, это ясно! Мне уже нет спасения! И ничего возвышенного нет в моей голове. Я задыхаюсь!\dots

Ой! Что же это такое? Сейчас что"=то произошло, но я не могу понять, что  именно. Я что"=то видел или что"=то слышал\dots

Ой! Опять что"=то произошло? Боже мой! Мне нечем дышать. Я, кажется, умираю\dots

А это еще что такое? Почему я пою? Кажется, у меня болит шея\dots Но где же сундук? Почему я вижу все, что находится у меня в комнате? Да никак я лежу на полу! А где же сундук?

Человек с тонкой шеей поднялся с пола и посмотрел кругом. Сундука нигде не было. На стульях и кровати лежали вещи, вынутые из сундука, а сундука нигде не было.

Человек с тонкой шеей сказал:

"--* Значит, жизнь победила смерть неизвестным для меня способом.

(В черновике приписка: жизнь победила смерть, где именительный падеж, а где винительный).

\begin{flushright}30 января 1937 года.\end{flushright}