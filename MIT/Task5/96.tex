\begin{flushleft}ВОСТРЯКОВ смотрит в окно на улицу:\end{flushleft}

Смотрю в окно и вижу снег.

Картина зимняя давно душе знакома.

Какой-то глупый человек

Стоит в подъезде противоположного дома.

Он держит пачку книг под мышкой

Он курит трубку с медной крышкой.

Теперь он быстрыми шагами

Дорогу переходит вдруг,

Вот он исчез в оконной раме.
           
\begin{center}(Стук в дверь).\end{center}

Теперь я слышу в двери стук.

Кто там?

\begin{flushleft}ГОЛОС ЗА ДВЕРЬЮ:\end{flushleft}
   
Откройте. Телеграмма.

\begin{flushleft}ВОСТРЯКОВ:\end{flushleft}

Врет. Чувствую, что это ложь.

И вовсе там не телеграмма.

Я сердцем чую острый нож.

Открыть иль не открыть?

\begin{flushleft}ГОЛОС ЗА ДВЕРЬЮ:\end{flushleft}

Откройте!

Чего вы медлите?

\begin{flushleft}ВОСТРЯКОВ:\end{flushleft}

Постойте!

Вы суньте мне под дверь посланье.

Замок поломан. До свиданья.

\begin{flushleft}ГОЛОС ЗА ДВЕРЬЮ:\end{flushleft}

Вам нужно в книге расписаться.

Откройте мне скорее дверь.

Меня вам нечего бояться,

Скорей откройте. Я не зверь.

\begin{flushleft}ВОСТРЯКОВ (приоткрывая дверь):\end{flushleft}

Войдите. Где вы? Что такое?

\begin{center}(Смотрит за дверь).\end{center}

Куда жеон пропал? Он не мог далеко уйти.

Спрятаться  тут негде. Куда же он делся?

Улица совсем пустая. Боже мой! И на снегу
   
нет следов! Значит, никто к моей двери не

подходил. Кто же стучал? Кто говорил со

мной через дверь?

\begin{center}(Закрывает дверь).\end{center}

\begin{flushright}[1937 - 1938 гг.]\end{flushright}