\documentclass{article}
\usepackage[T2A]{fontenc}
\usepackage[utf8]{inputenc}
\usepackage[russian]{babel}

\title{\textbf{ДАНИИЛ ХАРМС}}
\date{}

\begin{document}
\maketitle{}
Когда   жена  уезжает куда"=нибудь  одна, муж бегает по комнате и не находит себе места.

Ногти у мужа страшно  отрастают,  голова трясется, а лицо покрывается мелкими черными точками.

Квартиранты  утешают покинутого  мужа  и кормят его свиным зельцем. Но  покинутый муж теряет аппетит и преимущественно пьет пустой чай.

В это время его жена купается в озере и случайно задевает ногой подводную корягу. Из"=под коряги выплывает щука и кусает жену за пятку. Жена с криком выскакивает из воды и бежит к дому. Навстречу жене бежит хозяйская дочка. Жена показывает хозяйской дочке пораненную ногу и просит ее забинтовать.

Вечером жена пишет мужу письмо и подробно описывает свое злоключение.

Муж читает письмо и волнуется  до  такой степени, что роняет из рук стакан с водой, который падает на пол и разбивается.

Муж собирает осколки стакана и ранит ими себе руку.

Забинтовав пораненный палец, муж садится и пишет жене письмо. Потом выходит на улицу, чтобы бросить письмо в почтовую кружку.

Но на улице муж находит папиросную  коробку, а в коробке 30 000 рублей.

Муж  экстренно  выписывает жену обратно, и они начинают счастливую жизнь.

\begin{flushright}<\dots>\end{flushright}

\begin{center}* * *\end{center}

\begin{center}\subsection*{\textbf{СКАЗКА}}\end{center}

Жил"=был один человек, звали его Семенов.

Пошел  однажды  Семенов гулять и потерял носовой платок.

Семенов  начал  искать  носовой платок и потерял шапку.

Начал искать шапку и потерял куртку. Начал куртку искать и потерял сапоги.

"--* \textit{Ну,} "--* сказал Семенов, "--*  \textit{этак все растеряешь. Пойду лучше домой.} Пошел Семенов домой и заблудился.

"--* \textit{Нет,} "--* сказал Семенов, "--* \textit{лучше я сяду и посижу.} Сел Семенов на камушек и заснул.

\begin{flushright}<\dots>\end{flushright}

\begin{center}* * *\end{center}

\begin{center}\subsection*{\textbf{СЕВЕРНАЯ СКАЗКА}}\end{center}

Старик, не зная зачем, пошел в лес.  Потом вернулся и говорит: 

"--* \textit{Старуха, а старуха!} "--* Старуха так и повалилась. С тех пор все зайцы зимой белые.

\begin{center}* * *\end{center}

Одному  французу подарили диван, четыре стула и кресло.

Сел француз на стул у окна, а самому хочется на диване полежать.

Лег француз на диван, а ему уже на кресле посидеть хочется.

Встал француз с  дивана и сел на кресло, как король, а у самого  мысли  в голове  уже такие, что на кресле"=то больно пышно. Лучше попроще, на стуле.

Пересел француз на стул у окна, да только не сидится французу на этом стуле, потому что в окно как"=то дует.

Француз  пересел  на стул возле печки  и почувствовал, что он устал.

Тогда  француз решил лечь на диван и отдохнуть,  но, не дойдя до дивана, свернул в сторону и сел на кресло.

"--* \textit{Вот где хорошо!} "--* сказал француз, но сейчас же прибавил: "--* \textit{А на диване-то, пожалуй, лучше.}

\begin{flushright}<\dots>\end{flushright}

\begin{center}* * *\end{center}

\begin{flushleft}\hspace*{20mm}Жил на свете 

\hspace*{20mm}Мальчик Петя, 

\hspace*{20mm}Мальчик Петя Пинчиков. 

\hspace*{20mm}И сказал он: 

\hspace*{20mm}Тётя, тётя, 

\hspace*{20mm}Дайте, тётя, 

\hspace*{20mm}Блинчиков. 

\textit{}

\hspace*{20mm}Но сказала тётя Пете: 

\hspace*{20mm}Петя, Петя Пинчиков! 

\hspace*{20mm}Не люблю я, когда дети 

\hspace*{20mm}Очень клянчут блинчиков.\end{flushleft}

\begin{flushright}<1930-е>\end{flushright}

\begin{center}* * *\end{center}

\begin{center}\subsection*{\textbf{КИРПИЧ}}\end{center}

Господин  невысокого роста с камушком  в глазу подошел к двери табачной лавки и остановился. Его черные лакированные туфли сияли у каменной ступенечки,  ведущей  в  табачную лавку. Носки туфель были  направлены  внутрь магазина. Еще два шага, и господин скрылся бы за дверью.

Но он почему"=то задержался, будто нарочно для того, чтобы подставить голову под кирпич, упавший с крыши. Господин даже снял шляпу, обнаружив свой лысый череп, и таким образом кирпич ударил господина прямо по голой голове, проломил черепную кость и застрял в мозгу.

Господин не упал. Нет, он только пошатнулся от страшного удара, вынул из кармана платок, вытер им лицо\dots и, повернувшись к толпе, которая мгновенно собралась вокруг 
этого господина, сказал:

"--* \textit{Не беспокойтесь, господа,  у меня была уже  прививка.  Вы видите,  у меня в правом глазу торчит камушек?  Это тоже был однажды случай. Я уже привык к этому. Теперь мне все трын"=трава!} И с этими словами господин надел шляпу и ушел  куда-то  в сторону, оставив  смущенную толпу в полном недоумении.

\begin{flushright}<\dots>\end{flushright}

\begin{center}* * *\end{center}

\begin{center}\subsection*{\textbf{ПЬЕСА}}\end{center}

ШАШКИН (стоя посредине сцены): У меня сбежала жена. Ну что же тут поделаешь? Все равно, коли сбежала, так уж не вернешь. Надо быть философом и мудро  воспринимать всякое событие. Счастлив тот,  кто обладает  мудростью. Вот Куров этой мудростью не обладает, а я обладаю. Я в Публичной библиотеке два раза  книгу читал. Очень умно там обо  всем написано.

Я всем интересуюсь, даже языками. Я знаю по"=французски считать и знаю по-немецки  живот. Дер маген. Вот как! Со мной даже художник  Козлов  дружит.  Мы  с  ним вместе пиво пьем. А Куров что? Даже на часы  смотреть не умеет. В пальцы сморкается, рыбу вилкой ест, спит в сапогах, зубов не чистит\dots тьфу! Что
называется"--* мужик!  Ведь с ним  покажись  в обществе: вышибут вон, да еще и матом покроют"--*  не ходи, мол, с мужиком, коли сам интеллигент.

Ко  мне не подкопаешься.  Давай графа~--- поговорю с графом. Давай барона~--- и с бароном поговорю. Сразу даже не поймешь,  кто  я такой есть.

Немецкий язык, это я, верно, плохо знаю: живот~--- дер маген. А вот  скажут мне: <<Дер маген фин дель мун>>,~--- а  я уже и не знаю, чего это такое. А Куров тот и <<дер маген>> не знает. И ведь с таким дурнем  убежала! Ей, видите ли, вон чего надо! Меня она, видите ли, за мужчину не считает. <<\textit{У тебя,}~--- говорит,~--- \textit{голос бабий!}>> Ан и не бабий,  а  детский у меня голос! Тонкий, детский, а вовсе не бабий! Дура такая! Чего ей Куров дался? Художник  Козлов говорит, что с меня садись да картину пиши.

\begin{flushright}<\dots>\end{flushright}

\begin{center}* * *\end{center}

\begin{center}\subsection*{\textbf{ПРОИСШЕСТВИЕ НА УЛИЦЕ}}\end{center}

Однажды один человек соскочил с трамвая, да так  неудачно, что попал под автомобиль. Движение уличное остановилось, и милиционер принялся выяснять, как произошло несчастье. Шофер долго что"=то объяснял, показывая пальцами на передние колеса автомобиля. Милиционер ощупал эти колеса и записал в книжечку название улицы. Вокруг собралась довольно многочисленная толпа. Какой-то человек с тусклыми глазами все время сваливался с тумбы. Какая"=то дама все оглядывалась на другую даму, а та, в свою очередь, все оглядывалась на первую даму. Потом толпа разошлась и уличное движение восстановилось.

Гражданин с тусклыми  глазами еще долго сваливался с тумбы, но наконец и он, отчаявшись, видно, утвердиться на тумбе, лег просто на тротуар. В это время какой-то человек, несший стул, со всего  размаха  угодил  под трамвай. Опять пришел милиционер, опять собралась толпа и остановилось  уличное движение.  И гражданин с тусклыми глазами  опять начал сваливаться с тумбы.

Ну, а  потом опять все стало хорошо, и даже Иван Семенович Карпов завернул в столовую.

\begin{flushright}<\dots>\end{flushright}

\end{document}
