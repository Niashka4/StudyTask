\documentclass{article}
\usepackage[T2A]{fontenc}
\usepackage[utf8]{inputenc}
\usepackage{amsthm}
\usepackage{amsmath}
\usepackage{amssymb}
\usepackage{amsfonts}
\usepackage{mathrsfs}
\usepackage[12pt]{extsizes}
\usepackage{fancyvrb}
\usepackage{indentfirst}
\usepackage[
  left=2cm, right=2cm, top=2cm, bottom=2cm, headsep=0.2cm, footskip=0.6cm, bindingoffset=0cm
]{geometry}
\usepackage[russian,english]{babel}


\begin{document}

Производя подстановку для одновременного применения обоих сдвигов, получим:

\begin{equation}
\begin{cases}
A = 3 \delta - b_s = -b - 3 a \Delta + 3 \delta - 6 \Delta^2, \\
\begin{array}{lll}
B & = & 3 \delta^2 - 2 b_s \delta + a_s c_s - 4 d_s =\\
& = & 3 \delta^2 - 2(b + 3 a \Delta + 6 \Delta^2) \delta + (a + 4 \Delta)(c + 2 b \Delta + 3 a \Delta^2 + 4 \Delta^3) - \\
&   & - 4(d + c \Delta + b \Delta^2 + a \Delta^3 + \Delta^4), \\
C & = & \delta^3 - b_s \delta^2 + (a_s c_s - 4 d_s) \delta - a_s^2 d_s - c_s^2 + 4 b_s d_s = \\
& = & \delta^3 - (b + 3 a \Delta + 6 \Delta^2) \delta^2 + ((a + 4 \Delta)(c + 2 b \Delta + 3 a \Delta^2 + 4 \Delta^3) - \\
&   & - 4(d + c \Delta + b \Delta^2 + a \Delta^3 + \Delta^4)) \delta - (a + 4 \Delta)^2(d + c \Delta + b \Delta^2 + a \Delta^3 + \Delta^4) - \\
&   & - (c + 2 b \Delta + 3 a \Delta^2 + 4 \Delta^3)^2 + 4(b + 3 a \Delta + 6 \Delta^2)(d + c \Delta + b \Delta^2 + a \Delta^3 + \Delta^4). 
\end{array}
\end{cases}
\end{equation}

Расчёты показывают, что когда $\delta = \alpha a_s^2 + \beta b_s$, где $8 \alpha +3 \beta = 1$, коэффициенты резольвенты не меняются при любом сдвиге x.

\end{document}

